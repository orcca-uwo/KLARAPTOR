\chapter{The implementation of KLARAPTOR}
\label{ch:implementation}

%In the previous sections we gave an overview of our technique for
%general multithreaded programs $\cal{P}$. For the implementation of this
%technique, and the resulting experimentation and performance analysis,
%we focus on programs for GPU architectures using
%the programming model {\cuda}.
%These programs interleave 
%\begin{inparaenum}[(i)]
%  \item serial code which is executed
%on the host (the CPU) and, multithreaded
%code which is  executed on the device (the GPU).
%\end{inparaenum}
%The host launches a device code execution by calling a particular type
%of function, called a {\em kernel}.

This section is an overview of the implementation of
our previously presented technique (Section~\ref{sec:steps})
specialized to {\cuda} in the KLARAPTOR tool.
Our tool is built in the C language, 
making use of the LLVM Pass Framework (see Section~\ref{sec:io-builder}) 
and the NVIDIA Nsight Compute CLI (\ncu) (see Section~\ref{sec:data_collection}).
KLARAPTOR is freely available in source at 
\textcolor{navy}{\href{https://github.com/orcca-uwo/KLARAPTOR}{https://github.com/orcca-uwo/KLARAPTOR}}.

%%
%%
In the case of a {\cuda} kernel, the data parameters
specify the input data size.
In many examples this is a single parameter, say $N$,
describing the size of an array (or the order of a multi-dimensional array), 
the values of which are usually powers of $2$.
%%
Program parameters describe
the kernel launch parameters, i.e. grid and thread block dimension sizes, 
and are also typically powers of $2$.
For example, a possible thread block configuration
may be $1024 \times 1 \times 1$ (a one-dimensional thread block),
or $16 \times 16 \times 2$ (a three-dimensional thread block).
Lastly, the hardware parameters are values specific to the 
target GPU, for example, memory bandwidth, the number of SMs available,
and their clock frequency.
%In {\cuda} (for Compute Capability 2.x or higher) 
%the number of threads per block must be a multiple of 32 less than 
%1024. This limits the possible thread block configurations
%and thus the set of feasible program parameters. 

We organize this section as follows.
Sections~\ref{sec:annotation} and \ref{sec:io-builder}
are specific to our implementation and do not
correspond to any step of Section~\ref{sec:steps}.
The compile time steps 1 (data collection) and 2 (rational function estimation)
are reflected in Sections~\ref{sec:data_collection}
and \ref{sec:paramest}, respectively, while step 3 requires no explanation.
%%
The runtime steps 4 (rational program evaluation)
and 5 (program execution) are trivial to perform. 
Throughout this section,  the term {\em rational program}
refers to the mathematical concept defined in Section~\ref{sect:ratprog}
whereas the term {\em helper program} refers to the generated code
which implements rational programs in order to select kernel configurations.


%Throughout the current and the following sections, 
%our discussions make use of these specialized terms, 
%input data size and thread block configuration, for data and program parameters, 
%respectively, in order to make clear explanations and associations
%between theory and practice.

%In summary, our implementation proceeds as follows:
%\begin{enumerateshort}
%	\item[(2)] 
%	\item[(3)] A driver program orchestrates the combination of 
%	device-specific characteristics (i.e.\ a device profile) and various
%	configurations of program and data parameters to be passed to the instrumentor.
%	Running the instrumentor (via a profiler on a GPU) 
%	measures and records the required performance metrics.
%%	(to be used by the MWP-CWP model),
%%	and records the results for each kernel.
%%	It is important to note that the mentioned
%%	result for each kernel is the average of all of the kernel calls occurring in one
%%	run of the \cuda\ program.
%	
%%	\item[(4)] For each kernel, we use the program parameter configurations,
%%	together with the estimated execution time obtained from step (3)
%%	to perform parameter estimation and obtain our objective function(s).
%\end{enumerateshort}

%%%%%%%%%%%%%%%%%%%%%%%%%%%%%%%%%%%%%%%%%%%%%%%%%
%%%%%%%%%%%%%%%%%%%%%%%%%%%%%%%%%%%%%%%%%%%%%%%%%
\section{Annotating and preprocessing source code}
\label{sec:annotation}
	Beginning with a {\cuda} program written in C/C++, we minimally annotate the host code
	to make it compatible with our \textit{pre-processor}.
	%specifying the code fragment in the host code which calls the kernel.
%	- using pragmas to specify the dimensions of a kernel, and specifying the 
%	kernel input parameter which referes to the problem size. 
%	{\todo{EXAMPLE NEEDED HERE WITH THE ACTUAL ANNOTATION}{}}
	We take into account the following points:
	\begin{enumerateshort}
	\item[(i)] 
	the code targets at least CUDA Compute Capability (CC) 7.5;
	\item[(ii)]
	there should be no {\cuda} runtime API calls as such calls will interfere
	with later {\cuda} driver API calls used by our tool, for example, \texttt{cudaSetDevice};
%	for example, 
%	{\texttt{cudaSetDevice}} should not be used with the tool as it leads to undefined behavior;
	\item[(iii)]
	the block dimensions and grid dimensions must be declared 
	as the typical {\cuda} {\dimThree} structs.
%	 defined in {\texttt{"CUDA\_PATH/include/vector\_types.h"}}
\end{enumerateshort}

For each kernel in the {\cuda} code, we add two 
pragmas, one specifying the dimension 
of the kernel (1, 2, or 3), and one defining the index of the kernel input argument 
corresponding to the data size $N$.
For instance, consider the code segment below of a {\cuda} kernel
and added pragmas.
This kernel operates of a two-dimensional array of order $N$, 
making it a two-dimensional kernel.	
{
\begin{tcolorbox}
\scriptsize
\begin{verbatim}
#pragma kernel_info_size_param_idx_Sample = 1; 
#pragma kernel_info_dim_sample_kernel = 2;
__global__ void Sample (int *A, int N) { 
    int tid_x = threadIdx.x + blockIdx.x*blockDim.x;
    int tid_y = threadIdx.y + blockIdx.y*blockDim.y;
    ...
}
\end{verbatim}
\end{tcolorbox}
%\vspace{-2em}
}


Lastly, for each kernel, the user must fill two formatted configuration 
files which follow \texttt{Python} syntax.
One specifies the constraints on the thread block configuration
while the other specifies the grid dimensions.
%\texttt{kernel\_NAME\_gridconf.conf} and \texttt{kernel\_NAME\_constraints.conf},
%where \texttt{NAME} is the name of the kernel.
For example, for the 2D kernel \texttt{Sample} above, 
one could specify that its thread block configuration ($bx,by,bz$) must satisfy 
$bx < by^2$, $bx < N$ and $by < N$.
Since the kernel dimension is given as 2, we assume $bz=1$. 
Similarly, the grid dimensions ($gx,gy,gz$), could be specified as
$gx=\lceil \frac{N}{bx} \rceil$,  
$gy=\lceil  \frac{N}{by} \rceil$, $gz=1$. 
%The configuration files representing such are:
%\begin{tcolorbox}
%{\scriptsize
%\begin{verbatim}
%"kernel_Sample_constraints.conf":
%---------------------------------
%n - bx
%n - by
%bx - by**2 
%\end{verbatim}
%}
%%%
%\end{tcolorbox}		
%\begin{tcolorbox}
%{\scriptsize
%\begin{verbatim}
%"kernel_Sample_gridconf.conf":
%------------------------------
%gx=ceil(n/bx)
%gx=ceil(n/bx)
%gz=1
%\end{verbatim}
%}
%\end{tcolorbox}
	

%%%%%%%%%%%%%%%%%%%%%%%%%%%%%%%%%%%%%%%%%%%%%%%%%
%%%%%%%%%%%%%%%%%%%%%%%%%%%%%%%%%%%%%%%%%%%%%%%%%
%\subsection{Preprocessing}
%\label{sec:preprocessing}
%%%%%%%%%%%%%%%%%%%%%%%%%%%%%%%%%%%%%%%%%%%%%%%%%
Now, a preprocessor processes 
the annotated source code, replacing {\cuda} runtime API calls with driver 
API kernel launches. 
This step includes source code analysis in order to extract
a list of kernels, a list of kernel calls in the host code, 
and finally, the body of each kernel to be used for further analysis.
A so-called ``PTX lookup table'' is built to store
kernel information and static parameters.
This table will be inserted into the ``instrumented binary'', the
compiled {\cuda} program augmented by the helper programs. 
%\begin{enumerateshort}
%\item[(3)]
%In the meantime, a non-instrumented binary and the binary
%for the rational program itself are generated as well.

%The pre-processor program prepares the code for 
%	collecting kernel-specific runtime values.
	
	%%
	%Finally, the pre-processor program uses the specified CC 
	%in the host code to determine the architecture-specific 
	%performance counters for each kernel.
	%%
%	The result of this step is an executable file, which we will refer to as
%	\textit{the instrumentor}. The instrumentor takes as input the same program
%    parameters as the original {\cuda} code.
%\end{enumerateshort}


%%%%%%%%%%%%%%%%%%%%%%%%%%%%%%%%%%%%%%%%%%%%%%%%%
%%%%%%%%%%%%%%%%%%%%%%%%%%%%%%%%%%%%%%%%%%%%%%%%%
\section{Input/Output builder}
\label{sec:io-builder}
The Input/Output builder Pass, or IO-builder,
is a compiler pass written in the LLVM Pass Framework to
%% three descriptions which must be combined and simplified:
build the previously mentioned ``instrumented binary''.
This pass embeds an IO mechanism (i.e. a function call)
to communicate with the helper program of a kernel for each of its invocations.
%There is one such function call embedded per kernel call.
Thus, at the runtime of the {\cuda} program being analyzed (step 5 of Section~\ref{sec:steps}),
an IO function is called before each kernel invocation to return six integers,
$(gx, gy, gz, bx, by, bz)$, the optimal kernel launch parameters.
%The IO mechansim is based on "POSIX pipes" that avoid writing on disk
%and only use the main memory.
%From this point on, we only deal with intermediate representation (IR) 
%of the code in LLVM compiler infrastrucure. 

The IO-builder pass goes through the following steps:
\begin{enumerate}[(i)]
\item obtain the LLVM intermediate representation
of the instrumented source code and find all {\cuda} driver API kernel calls;
\item relying on the annotated information for each kernel,
determine which variables in the IR contain the value of $N$ for a 
corresponding kernel call; and
\item insert a call to an IO function immediately before each kernel call
in order to pass the runtime value of $N$ to the
corresponding helper program and retrieve the optimal kernel
launch parameters.
\end{enumerate}
 
%%%%%%%%%%%%%%%%%%%%%%%%%%%%%%%%%%%%%%%%%%%%%%%%%
%%%%%%%%%%%%%%%%%%%%%%%%%%%%%%%%%%%%%%%%%%%%%%%%%
\section{Building a helper program: data collection}
\label{sec:data_collection}
In order to perform the eventual rational
function approximation, 
we must collect data and statistics regarding certain performance counters and runtime
metrics (see \cite{DBLP:conf/isca/HongK09} and \cite{cuda2015}).
These metrics can be partitioned into three categories.

%\begin{enumerate}[(i)]
Firstly, \textit{architecture-specific performance counters} of a kernel,
characteristics influenced by the CC of the device.
These can be obtained at compile-time, since the target CC is specified at this time.
These characteristics include the number of registers used per thread,
the amount of static shared memory %(i.e. annotated in the code)
per thread block, and the number of (arithmetic and memory)
instructions per thread.
%This information can easily be obtained from 
%a {\cuda} compiler (e.g. NVIDIA's \texttt{NVCC}).
%passing source code of CUDA kernels to a 
%{\cuda} compiler (such as NVIDIA's \texttt{NVCC}).
%In this case, we rely on NVIDIA's \texttt{nvcc} compiler
%(the same can be achieved using \texttt{LLVM}).

Secondly, \textit{runtime-specific performance counters} that depend on
the behavior of the kernel at runtime.
This includes values impacted by memory access patterns, namely, 
%the number of coalesced and non-coalesced memory accesses per warp,
%the number of memory instructions in each thread that cause
%coalesced and non-coalesced accesses, 
%and eventually, the total number of warps that are being executed.
the number of memory accesses per warp, the number of memory instructions of each thread,
and the total number of warps that are being executed.
%%%%%%
%%%%%%
We have developed a customized profiler using NVIDIA's Nsight Compute CLI\
to collect the required runtime performance counters. 

Thirdly, \textit{device-specific parameters}, which
describe an actual GPU card, allow us to compute
a more precise performance estimate.
A subset of such parameters can be determined by microbenchmarking 
the device (see \cite{DBLP:journals/tpds/MeiC17} and \cite{DBLP:conf/ispass/WongPSM10}),
this includes the memory bandwidth, 
and the departure delay for memory accesses.
The remaining parameters can easily be obtained
by consulting the vendor's guide \cite{cuda2019guide}, 
%vendor's specification sheet for the device, 
or by querying the device itself via the {{\cuda} driver API}.
%in order to extract such metrics.
Such parameters include the number of SMs on the card, 
the clock frequency of SM cores, and the instruction delay.
%Such metrics can easily be obtained by specifying a particular GPU card.
%In practice, it suffices to perfom this step only once per each target GPU.

%%%%%%%%%%%%%%%%%%%%%%%%%%%%%%%%%%%%%%%%%%%%%%%%%
%%%%%%%%%%%%%%%%%%%%%%%%%%%%%%%%%%%%%%%%%%%%%%%%%
\section{Building a helper program: outlier removal}
\label{sec:outlier_removal}
As mentioned in Section~\ref{sec:contributions}, we now describe the outlier removal step,
which is performed by quartile fencing algorithm. 
The rationale behind integrating this algorithm lies in addressing potential 
noise in the empirical data gathered from NVIDIA's Nsight Compute (ncu), which could stem 
from factors such as dynamic voltage and frequency scaling (DVFS) and other variations in 
GPU performance. By pinpointing and eliminating upper outliers from the dataset, our 
objective is to enhance the accuracy of the subsequent parameter estimation process.

The quartile fencing algorithm is a robust statistical method for detecting and removing 
outliers from a dataset. It is based on the concept of interquartile range (IQR), which is 
the difference between the first quartile (Q1) and the third quartile (Q3) of the data. The 
algorithm defines outlier boundaries, called the lower and upper inner fences, by extending 
the IQR beyond Q1 and Q3\cite{everitt2010cambridge}:

\begin{enumerate}[(i)]
    \item $IQR = Q3 - Q1$.
    \item $UIF = Q3 + 1.5 \times IQR$.
\end{enumerate}

Data points falling outside the upper inner fence are considered as outliers and are removed from the 
dataset. To implement the outlier removal process, we first profile our program ${\cal P}$ for 
small input sizes of N and obtain the MWP-CWP estimation for clock-cycles for various thread block configurations. 
Next, we calculate the first and third quartiles (Q1 and Q3) along with the interquartile range 
(IQR) for the estimated clock-cycles. Using the quartile fencing algorithm, we determine the 
upper inner fence and identify the thread block configurations exceeding this threshold as outliers.

Once the outliers are identified, we remove them from the dataset before proceeding with the parameter
estimation process. This preprocessing step helps minimize the impact of upper outliers on the data 
and enhances the accuracy of the resulting parameter estimation. 

\section{Building a helper program: rational function approximation}
\label{sec:paramest}

Using the runtime data collected in the previous step, 
we look to determine the rational functions
$\hat{g}_i(\bm{D},\bm{P})$ (see Section~\ref{sec:steps})
which estimate the low-level metrics or other intermediate 
values in the rational program ${\cal R}$. For ease of
discussion, we replace the parameters $\bm{D}$ and $\bm{P}$
with the generic variables $X_1,\ldots,X_n$.

A rational function is simply a fraction of two polynomials: 
%\scalebox{0.88}{
%	\begin{minipage}{0.9\linewidth}
%		\begin{equation}
{
	\small
		\begin{align}
		f(X_1,\dots,X_n) &= \frac{\alpha_1\cdot(X_1^0\cdots X_n^0) \;+\; 
			\dots \;+\;  \alpha_i\cdot(X_1^{u_1}\cdots X_n^{u_n})}{\beta_1\cdot(X_1^0\cdots X_n^0) \;+\;
			\dots \;+\; \beta_j\cdot(X_1^{v_1}\cdots X_n^{v_n})}
		\end{align}
}
%		\end{equation}
%	\end{minipage}
%}
%\noindent
With a \textit{degree bound} (an upper limit on the exponent) on each
variable $X_k$ in the numerator and the denominator,
$u_k$ and $v_k$, respectively, these polynomials can be defined up to
some \textit{parameters} (using the language of parameter estimation), namely the coefficients of the polynomials,
$\alpha_1,\dots,\alpha_i$ and $\beta_1,\dots,\beta_j$.
Through algebraic analysis of performance models 
like the MWP-CWP model, and empirical evidence, these degree bounds are relatively small.

We perform a parameter estimation (for each rational function)
on the coefficients $\alpha_1, \ldots, \alpha_i, \beta_1, \ldots, \beta_j$
to determine the rational function precisely. 
This is a simple linear regression 
which can be solved by an over-determined
system of linear equations,
say by the method of linear least squares.
%
%
%
%The system of linear equations defined by 
%the model-fitting parameters $\alpha_1, \ldots, \alpha_i, \beta_1, \ldots, \beta_j$
%can very classically be defined as an equation
%of the form  $\mathbf{Ax} = \mathbf{b}$,
This system of linear equations is often described in matrix format  $\mathbf{Ax} = \mathbf{b}$,
where $\mathbf{A}$ (the \textit{sample matrix} or \textit{design matrix}) and 
$\mathbf{b}$ (the right-hand side vector)
encode the collected data, while the solution vector 
$\mathbf{x}$ encodes the 
model-fitting parameters. $\mathbf{A}$ being derived from a rational function 
implies that it is essentially the sample matrix for the denominator polynomial appended to
the sample matrix for the numerator polynomial\footnote{Keen observers 
will notice that, for rational functions,
we must actually solve a system of homogeneous equations.
Such details are omitted here, but we refer the reader to \cite[Chapter 5]{brandt2018high}.}.

Many different methods exist for solving this so-called linear least squares problem, 
such as the \textit{normal-equations}, or \textit{QR-factorization}, 
however, these methods are either numerically unstable (normal-equations), or will fail
if the sample matrix is rank-deficient (both normal-equations and QR) 
\cite{corless2013graduate}.
We rely then on the \textit{singular value decomposition} (SVD) of $\mathbf{A}$ 
to solve this problem.
%Given a matrix $\mathbf{A} \in \mathbb{R}^{m\times n}$, a singular value decomposition 
%of $\mathbf{A}$ is a factorization into the form $\mathbf{UDV}^T$ where 
%$\mathbf{U} \in \mathbb{R}^{m\times m}$ and $\mathbf{V} \in \mathbb{R}^{n \times n}$ are orthonormal, and $\mathbf{D} \in \mathbb{R}^{m \times n}$ is a (possibly rank deficient) diagonal matrix.
This decomposition is very computationally intensive, much more than that of
normal-equations or QR, but is also much more numerically
stable, as wel as being capable of producing solutions with a rank-deficient sample matrix.
%Moreover, it is capable of producing solutions even with a rank-deficient sample matrix.
%By taking advantage of the truncated singular value decomposition
%we gain evmen further stability when $\mathbf{A}$ is numerically rank-deficient \cite{hansen1990truncated}.

We are highly concerned with the robustness of our
method due to three problems present in our particular situation:
\begin{enumerate}
	\item[(1)] the sample matrix is very ill-conditioned;
	\item[(2)] the sample matrix will often be (numerically) rank-deficient;
	\item[(3)] we are interested in using our fitted model for extrapolation, meaning 
	any numerical error in the model fitting will grow very quickly \cite{corless2013graduate}.
\end{enumerate}
While (3) is an issue inherent to our model fitting problem, (1) and (2) result
from our choice of model, and how the sample points $(X_1,\ldots, X_n)$ are chosen, respectively.
Using a rational function (or polynomial) as the model for which we wish to estimate parameters
presents numerical problems. The resulting sample matrix is essentially 
%(or exactly, in the case of a polynomial model)
a Vandermonde matrix. These matrices,
while theoretically of full rank, are extremely ill-conditioned
%having exponentially 
%increasing condition numbers (a measure of its sensitivity to noise) 
\cite{corless2013graduate, beckermann2000condition}. 

Refering to (2), we discuss the difficulties in obtaining poised sample points for 
modeling functions that involve CUDA thread block dimensions as variables.
The geometric constraints, combined with the requirement that the product of dimensions be a multiple of 32, 
makes it challenging to achieve a full-rank sample matrix, which is crucial for accurate and stable model 
fitting. As a result, there is a higher likelihood of encountering 
a rank-deficient sample matrix \cite{chung1977lattices, olver2006multivariate}, which complicates the entire model estimation process.



%An exact solution is rarely defined if $m < n$ (where an infinite number of solutions is 
%possible) or if $m > n$. Therefore, we wish to 
%get a solution in the ``least squares sense'', that is, find 
%$\mathbf{x}$ such that \textit{residual} is minimized: 
%\begin{gather*}
%\mathbf{x} \:=\: \min_{\mathbf{x}}||\mathbf{r}||^2_2 \:=\: \min_{\mathbf{x}}||\mathbf{b} - \mathbf{Ax}||_2^2
%\end{gather*} 
%
%
%A typical solution to the linear least squares problem, and over-determined systems in general, 
%employs the \textit{QR-factorization}
%of the sample matrix $\mathbf{A}$
%\cite[Chapter 23]{gentle2012handbook}.
%However, our system is constructed from the evaluation 
%of monomial terms, resulting in essentially a Vandermonde matrix.
%Such matrices are very ill-conditioned.
%Since we are interested in using our fitted model for 
%extrapolation (i.e. estimating program parameters for new data parameters)
%any numerical errors in the model fitting will 
%grow very quickly \cite{corless2013graduate}. 
%Thus, our solution must be as numerically stable
%as possible. 
%However, since we have constructed the system of equations
%from monomial evaluations, it suffers from

Despite all of these challenges our parameter estimation techniques are well-implemented
in optimized C code. We use optimized algorithms from LAPACK (Linear Algebra PACKage) \cite{userguide:lapack}
for singular value decomposition and linear least squares solving
while rational function and polynomial implementations are similarly highly optimized
thanks to the Basic Polynomial Algebra Subprograms (BPAS) library \cite{bpasweb, brandt2018high}.
With parameter estimation complete the rational functions
required for the rational program are fully specified
and we can finally construct it.

\section{Helper programs}

In practice, the use of helper programs is split into two parts:
the generation of the rational program at the compile-time of the multithreaded 
program $\cal{P}$, 
and the use of the helper program during the runtime of $\cal{P}$.

\subsection{Compile-time code generation}
We are now at Step 3 of Section~\ref{sec:steps}.  We look to define a
helper program which evaluates the high-level metric $\cal{E}$ of the
program $\cal{P}$ using the \mwpcwp\ model.  In implementation, this
is achieved by using a previously defined \textit{helper program
template} which contains the formulas and case discussion of
the \mwpcwp\ model, independent of the particular program being
investigated. Using simple regular expression matching and text
manipulation we combine the helper program template with the rational functions 
determined in the previous step to obtain a helper program
specialized to the multithreaded program $\cal{P}$. The generation of
this helper program is performed completely during compile-time,
before the execution of the program itself.

\subsection{Runtime optimization}
At runtime, the input data sizes (data parameters) are well known.
In combination with the known hardware parameters, since the program
is actually being executed on a specific device,
we are able to specialize every parameter in the helper program
and obtain an estimate for the high-level metric $\cal{E}$.
This helper program is then easily and quickly evaluated during
(or immediately prior to) the execution of $\cal{P}$. Evaluating 
the helper program for each possible thread block configuration,
as restricted by our data parameters
and the \cuda\ programming model itself, we determine a thread block configuration
which optimizes $\cal{E}$. The program $\cal{P}$ is finally executed 
using this optimal thread block configuration. 
Therefore, we have completed Steps 4 and 5 of Section~\ref{sec:steps}.
%In practice, the use of rational programs is split into two parts:
%the generation of the rational program at the compile-time of 
%the multithreaded program 
%$\cal{P}$, 
%and the use of the rational program during the runtime of $\cal{P}$.
%%%%%%%%%%%%%%%%%%%%%%%%%%%%%%%%%%%%%%%%%%%%%%%%%
%%%%%%%%%%%%%%%%%%%%%%%%%%%%%%%%%%%%%%%%%%%%%%%%%

