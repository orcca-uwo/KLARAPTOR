\toremove{\section{Runtime Selection of Thread Block Configuration for a CUDA Kernel}}
\label{sec:heuristics}

\todo{remove this entire section.}{}

In the previous section we examined the general six-step process
to build and use a rational program to select program parameters. 
We now look to specialize this process to {\cuda} 
and describe precisely the algorithm followed by KLARAPTOR
to select program parameters (i.e. thread block configurations)
in its attempt to minimize kernel execution time (steps 4 and 5 in the general six-step process).

Here, our high-level performance metric ${\cal E}$ is 
execution time, and we make use of the {\mwpcwp} model
\cite{DBLP:conf/isca/HongK09, DBLP:conf/ppopp/SimDKV12}
to estimate it.
In this model the execution time is estimated as the estimated 
number of clock cycles {\ec}.
From the previous six-step process, it seems
trivial to simply select the configuration which 
minimizes {\ec}.
%At first, picking a subset of configurations with the lowest values of {\ec}
%seems trivial and we would expect it to work well.
However, in some practical examples, this can lead to a very poor choice
of thread block configuration which does not minimize execution time.
%
%
While the {\mwpcwp} model is a quite comprehensive performance prediction model, 
it mainly relies on occupancy for estimating the number of clock cycles. 
This has a serious drawback; the calculated occupancy is an upper
bound which can be too optimistic and is not a strong predictor of 
performance for compute-intensive kernels
(see \cite{Volkov:EECS-2016-143} and "Help" sheet of NVIDIA occupancy calculator spreadsheet, 
under "Notes on occupancy" \cite{nvidia:occupancy-calculator}).
Therefore, we still make use of {\mwpcwp} but do not 
take {\ec} directly as the performance predictor. 

Using the values of some low-level metrics (also defined within the {\mwpcwp} model),
together with {\ec},
we perform a case discussion in order to select a thread block configuration.
At runtime, the particular value of our data parameters is known,
%specifically the parameter $N$, whose value describes the size of an array 
%(or the order of a multi-dimensional array).
The values for the low-level metrics are then 
obtained using the data parameter values and the set of practically meaningful 
thread block configurations by evaluating
the previously obtained rational functions $\hat{g}_i(\bm{D}, \bm{P})$ 
for them.
This yields a dictionary whose keys are configurations
and values are a list of estimated performance metrics.
%%
The low-level metrics of interest are:
%%%%
%Note that we only interpolate and evaluate a subset 
%MWP-CWP of parameters including
\begin{inparaenum}[(i)]
\item the average number of computational instructions per thread ({\compinst}), 
\item the average number of memory instructions per thread ({\meminst}),
\item the average number of clock cycles spent on global memory transactions ({\memld}), and
\item the amount of dynamic shared memory used.% ({\texttt{dynamic-shared-mem-bytes}}).
\end{inparaenum}
We then also compute occupancy and {\ec} for each configuration.
%Naturally, we have multiple parameters to rely on for making a decision.
%%

\iffalse
Another strategy would be to pick a subset of configurations with the 
highest estimated occupancy. This approach can also lead to poor 
results due to the reasons explained for the previous strategy.
Therefore, we cannot consider this strategy as well.
%%Not really needed
\fi

At this point, we look to define a strategy which chooses 
a subset of configurations where any in the set 
would give near-optimal performance.
%which should minimize kernel execution time. 
Our proposed strategy takes into account not only occupancy
and {\ec}, but also the arithmetic intensity and efficiency of global memory
read/write transactions of a kernel.
We define arithmetic intensity as the ratio of 
{{\compinst}/{\meminst}}.
% for each block configuration in the dictionary.
%As we will see in {Section~\ref{sec:performance}}, 
%this approach yields promising results.
%%%%
The main idea is first to determine if the 
kernel is memory intensive or compute intensive.
{If the kernel is memory intensive, we look to minimize memory instructions, 
otherwise it is assumed to be compute intensive and we look to both minimize
time spent on memory transactions and maximize computational instructions. 
The latter might increase the chance of exploiting 
instruction-level parallelism (ILP) by the hardware 
(however, this is not guaranteed to happen at runtime).}
This strategy is detailed in Algorithm~\ref{alg-heuristic}.


%%%%%%%%%%%%%%%%%%%%%%%%%%%%%%%%%%%%%%%%%%%%%%%%%%
%%%%%%%%%%%%%%%%%%%%%%%%%%%%%%%%%%%%%%%%%%%%%%%%%%
%\begin{algorithm}[H]
%    \caption{HEURISTIC}
%    \label{alg-heuristic}
%	{\small
%    \begin{algorithmic}[1] % The number tells where the line numbering should start
%        \Procedure{HEURISTIC} {Dictionary {\confDict}, 
%        M as the memory efficiency percentage of the kernel.} 
%      	%%%
%        \State {[{\nparts}, {\nrepeat}] $\gets$ [10, number of entries in {\confDict}]};
%%        \State {{\param} $\gets$ "ArithmeticIntensity"};
%        \State {\confDict $\gets$ STABILIZE ({\confDict}, "ArithmeticIntensity", {\nparts}, {\nrepeat});}
%    	\State \Comment {Stabilize {\confDict} w.r.t. Arithmetic Intensity.}
%        %%%%
%	%- repeat for the number of elements in the config list.
%	%- the number of bins is set to 10, why 10?
%		\State {AAR $\gets$	 Average of {arithmetic intensity} for entries of {\confDict}}.
%	\State 	{MET $\gets$ 25\%} \Comment{Setting a memory efficiency threshold to 25\%.}
%    %%%
%	\If { M <= MET and AAR $\leq 1.0$}\Comment {Memory-intensive}
%	\State {[{\nparts}, {\nrepeat}] $\gets$ [10, 10]};
%	\State{\confDict $\gets$ Stabilize ({\confDict}, "clock cycles", {\nparts}, {\nrepeat});}
%	\State \Comment {Stabilize {\confDict} w.r.t. "Clock Cycles".}
%    %%% 
%	\State {[{\nparts}, {\nrepeat}] $\gets$ [10, 1]};
%	\State{\confDict $\gets$ Stabilize ({\confDict}, "{\meminst}", {\nparts}, {\nrepeat});}
%	\State \Comment {Stabilize {\confDict} w.r.t. "{\meminst}".}
%    %%%
% 	\State {[{\nparts}, {\nrepeat}] $\gets$ [10, 1]};
%	\State{\confDict $\gets$ GetExtreme ({\confDict}, "MIN", "{\meminst}", {\nparts}, {\nrepeat});}
%	\State \Comment {Get the partition with lowest values of "{\meminst}".}
%    %%%	    
%	\State {[{\nparts}, {\nrepeat}] $\gets$ [10, 1]};
%	\State{\confDict $\gets$ Stabilize ({\confDict}, "{\compinst}", {\nparts}, {\nrepeat});} 
%	\State \Comment {Stabilize {\confDict} w.r.t. "{\compinst}".}
%    %%%
%	\Else \Comment{Compute-intensive}
%	\State {[{\nparts}, {\nrepeat}] $\gets$ [4, 1]};
%	\State{\confDict $\gets$ Stabilize ({\confDict}, "{occupancy}", {\nparts}, {\nrepeat});}
%	\State \Comment {Stabilize {\confDict} w.r.t. "{occupancy}".}
%    %%% - set into 4 categories to cover 25~100% occupancy.        
%    %%%
% 	\State {[{\nparts}, {\nrepeat}] $\gets$ [10, 1]};
%	\State{\confDict $\gets$ GetExtreme ({\confDict}, "MIN", "{\memld}", {\nparts}, {\nrepeat});}
%	\State \Comment {Get the partition with lowest values of "{\memld}".}
%    %%%
% 	\State {[{\nparts}, {\nrepeat}] $\gets$ [10, 1]};
%	\State{\confDict $\gets$ GetExtreme ({\confDict}, "MAX", "{\compinst}", {\nparts}, {\nrepeat});}
%	\State \Comment {Get the partition with highest values of "{\compinst}".}
%    %%% 
%	\State {[{\nparts}, {\nrepeat}] $\gets$ [10, 1]};
%	\State{\confDict $\gets$ Stabilize ({\confDict}, "{\meminst}", {\nparts}, {\nrepeat});}
%	\State \Comment {Stabilize {\confDict} w.r.t. "{\meminst}".}
%	\EndIf
%	\State {Return {\confDict}};
%    \EndProcedure
%    \end{algorithmic}
%    }
%\end{algorithm}
%%%%%%%%%%%%%%%%%%%%%%%%%%%%%%%%%%%%%%%%%%%%%%%%%
%%%%%%%%%%%%%%%%%%%%%%%%%%%%%%%%%%%%%%%%%%%%%%%%%
\begin{algorithm}[!htb]
    \caption{HueristicSelection}
    \label{alg-heuristic}
	{\small
    \begin{algorithmic}[1] % The number tells where the line numbering should start
        \Procedure{HueristicSelection} {Dictionary {\confDict} of [key,value] pairs
        with block configurations as keys and values of performance metrics as
        values,
         {\memEfficiency} as average memory efficiency (percentage) of the kernel} 
      	%%%
        \State {[{\nparts}, {\nrepeat}] $\gets$ [10, number of entries in {\confDict}]};
%        \State {{\param} $\gets$ "ArithmeticIntensity"};
        \State {\confDict $\gets$ Stabilize ({\confDict}, "ArithmeticIntensity", {\nparts}, {\nrepeat});}
    	\State \Comment {Stabilize {\confDict} w.r.t. values of Arithmetic Intensity.}
        %%%%
	%- repeat for the number of elements in the config list.
	%- the number of bins is set to 10, why 10?
		\State {{\texttt{AAR}} $\gets$	 Average of {arithmetic intensity} for entries of {\confDict}}.
	\State 	{{\texttt{MET}} $\gets$ 25\%} \Comment{Setting the memory efficiency threshold to 25\%.}
    %%%
	\If { {\memEfficiency} <= {\memThreshold} and {\texttt{AAR}} $\leq 1.0$}\Comment {Memory-intensive}
	\State {[{\nparts}, {\nrepeat}] $\gets$ [10, 10]};
	\State{\confDict $\gets$ Stabilize ({\confDict}, "{\ec}", {\nparts}, {\nrepeat});}
%	\State \Comment {Stabilize {\confDict} w.r.t. "Clock Cycles".}
    %%% 
	\State {[{\nparts}, {\nrepeat}] $\gets$ [10, 1]};
	\State{\confDict $\gets$ Stabilize ({\confDict}, "{\meminst}", {\nparts}, {\nrepeat});}
%	\State \Comment {Stabilize {\confDict} w.r.t. "{\meminst}".}
    %%%
 	%\State {[{\nparts}, {\nrepeat}] $\gets$ [10, 1]};
	\State{\confDict $\gets$ Optimize ({\confDict}, "MIN", "{\meminst}", {\nparts}, {\nrepeat});}
%	\State \Comment {Get the subset with lowest values of "{\meminst}".}
    %%%	    
	%\State {[{\nparts}, {\nrepeat}] $\gets$ [10, 1]};
	\State{\confDict $\gets$ Stabilize ({\confDict}, "{\compinst}", {\nparts}, {\nrepeat});} 
%	\State \Comment {Stabilize {\confDict} w.r.t. "{\compinst}".}
    %%%
	\Else \Comment{Compute-intensive}
	\State {[{\nparts}, {\nrepeat}] $\gets$ [25, 1]};
	\State{\confDict $\gets$ Stabilize ({\confDict}, "{Occupancy}", {\nparts}, {\nrepeat});}
%	\State \Comment {Stabilize {\confDict} w.r.t. "{occupancy}".}
    %%% - set into 4 categories to cover 25~100% occupancy.        
    %%%
 	\State {[{\nparts}, {\nrepeat}] $\gets$ [10, 1]};
	\State{\confDict $\gets$ Optimize ({\confDict}, "MIN", "{\memld}", {\nparts}, {\nrepeat});}
%	\State \Comment {Get the partition with lowest values of "{\memld}".}
    %%%
 	%\State {[{\nparts}, {\nrepeat}] $\gets$ [10, 1]};
	\State{\confDict $\gets$ Optimize ({\confDict}, "MAX", "{\compinst}", {\nparts}, {\nrepeat});}
%	\State \Comment {Get the partition with highest values of "{\compinst}".}
    %%% 
	%\State {[{\nparts}, {\nrepeat}] $\gets$ [10, 1]};
	\State{\confDict $\gets$ Stabilize ({\confDict}, "{\meminst}", {\nparts}, {\nrepeat});}
%	\State \Comment {Stabilize {\confDict} w.r.t. "{\meminst}".}
	\EndIf
	\State {Return {\confDict}};
    \EndProcedure
    \end{algorithmic}
    }
\end{algorithm}
%%%%%%%%%%%%%%%%%%%%%%%%%%%%%%%%%%%%%%%%%%%%%%%%%
%%%%%%%%%%%%%%%%%%%%%%%%%%%%%%%%%%%%%%%%%%%%%%%%%
%\begin{algorithm}[!htb]
%    \caption{Stabilize}
%    \label{alg-stabilize}
%    \begin{algorithmic} % The number tells where the line numbering should start
%        \Procedure{Stabilize}{Dictionary {\confDict}, {\param}, {\nparts}, {\nrepeat}}
%		\State{{\texttt{stdDev}} $\gets$ -1};
%		\For {$(i=0; i<{\nrepeat}; i{\texttt{+=}1})$ }
%			\State{{\texttt{currentDev}} $\gets$ ComputeStandardDeviation({\confDict}, {\param})};
%			\State\Comment{Compute standard deviation for values of "{\param}" in {\confDict}.}
%			\If {{\texttt{currentDev}}=={\texttt{stdDev}}:}
%				\State {Break;}
%%				\Comment{Stop as soon as the standard deviation does not change.}
%			\EndIf;
%		\State{{\texttt{stdDev}}$\gets$ {\texttt{currentDev}}};
%		\State {{\confDict} $\gets$ RemoveExtreme({\confDict}, "MIN", {\param}, {\nparts});}
%		\EndFor
%		\State {Return {\confDict}};
%        \EndProcedure
%    \end{algorithmic}
%\end{algorithm}
%%%%%%%%%%%%%%%%%%%%%%%%%%%%%%%%%%%%%%%%%%%%%%%%%%
%%%%%%%%%%%%%%%%%%%%%%%%%%%%%%%%%%%%%%%%%%%%%%%%%%
%
%\begin{algorithm}[!htb]
%    \caption{GetExtreme}
%    \label{alg-get-extreme}
%    \begin{algorithmic} % The number tells where the line numbering should start
%        \Procedure{GetExtreme}{Dictionary {\confDict}, END, {\param}, {\nparts}}
%       	\State {$[S_0, \cdots, S_{\nparts-1}]$ $\gets$ 
%        	Split {\confDict} into {\nparts} subsets w.r.t. values of {\param}.}
%        \If {END $==$ "MIN"}
%        	\State {\confDict $\gets$ $S_0$}.
%			\Comment{Only keep $S_0$.}
%		\Else \Comment {END == "MAX"}
%        	\State {\confDict $\gets$ $S_{\nparts-1}$}.
%			\Comment{Only keep $S_{\nparts-1}$.}
%		\EndIf
%		\State {Return {\confDict}};
%        \EndProcedure
%    \end{algorithmic}
%\end{algorithm}
%%%%%%%%%%%%%%%%%%%%%%%%%%%%%%%%%%%%%%%%%%%%%%%%%%
%%%%%%%%%%%%%%%%%%%%%%%%%%%%%%%%%%%%%%%%%%%%%%%%%%
%\begin{algorithm}[!htb]
%    \caption{RemoveExtreme}
%    \label{alg-remove-extreme}
%    \begin{algorithmic} % The number tells where the line numbering should start
%        \Procedure{RemoveExtreme}{Dictionary {\confDict}, END, {\param}, {\nparts}}
%       	\State {$[S_0, \cdots, S_{\nparts-1}]$ $\gets$ 
%        	Split {\confDict} into {\nparts} subsets w.r.t. values of {\param}.}
%        \If {END $==$ "MIN"}
%        	\State {\confDict $\gets$ $[S_1, \cdots, S_{\nparts-1}]$}.
%        	\Comment{Remove $S_0$.}
%		\Else \Comment {END == "MAX"}
%        	\State {\confDict $\gets$ $[S_0, \cdots, S_{\nparts-2}]$}.
%			\Comment{Remove $S_{\nparts-1}$.}
%		\EndIf
%		\State {Return {\confDict}};
%        \EndProcedure
%    \end{algorithmic}
%\end{algorithm}


Certain subroutines are used within Algorithm~\ref{alg-heuristic}
in an attempt to filter outliers and select candidate configurations.
The first function, \texttt{Stabilize}, removes outliers
by iteratively filtering out configurations.
The iteartion proceeds until the standard deviation of the values
of the target metric no longer changes, or a maximum
number of iterations is reached.
To remove outliers, configurations whose value of the chosen metric 
falls in the top or bottom X\% of values 
are removed.
Note that standard deviation should indeed eventually reach some
fixed value since the metrics take finitely many values.
The next function, \texttt{Optimize}, simply gets
the subset of configurations whose value for a particular 
metric falls in the top X\% (if ``MAX'' is specified) 
or the bottom X\% (if ``MIN'' is specified).

%Algorithms \ref{alg-stabilize}, \ref{alg-get-extreme}, and \ref{alg-remove-extreme} are
%used as the building blocks of Algorithm \ref{alg-heuristic}.
%Algorithm \ref{alg-stabilize} tries to eliminate outlier configurations in the dictionary.
%Also, Algorithm~\ref{alg-get-extreme} (and \ref{alg-remove-extreme})
%filters out (and removes) the block configurations in the dictionary 
%which have the lowest or highest values for a specific parameter.
%at one end of extreme (with the lowest, or highest values) for a specific parameter.

%In summary, if the kernel is memory intensive:
%	find the points with most represented CC, most represented value of mem-inst,
%	pick the family with the lowest mem-inst, pick the one with most represented comp-inst,
%	and finally, get the family with highest mem-inst.
%	
%In summary, if the kernel is compute intensive:
%	Find the points that have most represented occupancy value, have the
%	lowest memory delay, perform the largest number of comp instructions, and
%	finally, stabilize on memory inst.


%%%%%
To decide whether a kernel is memory intensive or not, 
first, using a profiler (through {\nvprof} or {\cupti}), 
we measure  the average value of global memory (load/store) 
efficiency for the kernel for a set of block configurations.
%launch that %(to ensure how the kernel accesses memory at its best)
This value, which we will refer to as "Average Memory Efficiency" ({\memEfficiency})
should be less than a certain threshold. This threshold, which we will refer to
as "Memory Efficiency Threshold" ({\memThreshold})
can be determined through empirical study.
Currently, we set {\memThreshold} to 25\%.
Using {\memEfficiency} together with the mean arithmetic intensity 
of all configurations, indicates if a kernel is memory intensive or not.
%Although critical, however, this metric alone is not enough to determine memory 
%intensity. 
%%%
In particular, the mean arithmetic intensity should be at most equal to 1, 
that is, for every memory instruction there is at most one computational
instruction performed. 
%For a kenrel to be memory intense we must also have low memory efficiency.
%
Finally, we note that the values for {\nparts} and {\nrepeat} in Algorithm~\ref{alg-heuristic} 
have also been determined through experimentation (Lines 2, 8, 10, 16, and 18).
