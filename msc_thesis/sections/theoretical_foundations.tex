\chapter{Theoretical Foundations}
\label{ch:foundations}



Let ${\cal P}$ be a multithreaded program to be executed on a specific 
multiprocessor.
Parameters influencing  the performance of ${\cal P}$ include: 
\begin{inparaenum}[(i)]
\item {\em data parameters}, specifying the size and possibly
  structural characteristics of the data, 
\item {\em hardware parameters}, specifying
  characteristics of hardware resources, and
\item {\em program parameters}, specifying
how   work (e.g. threads) is mapped to
  hardware resources.
\end{inparaenum}
%% 	
By fixing the target architecture, the hardware parameters, say, ${\bm
  H} = \left(H_1, \ldots, H_h\right)$ become fixed and we can assume
that the performance of ${\cal P}$ depends only on data parameters
${\bm D} = \left(D_1, \ldots, D_d\right)$ and program parameters ${\bm
  P} = \left(P_1, \ldots, P_p\right)$.
%%
\ifoldversion Moreover, an optimal choice of ${\bm P}$ naturally
depends on a specific choice of ${\bm D}$.  \fi
%%
For example, in programs targeting GPUs 
%(e.g. programs written in {\cuda}),   %\ or OpenCL programs
the parameters ${\bm D}$ are typically dimension sizes of data
structures, like arrays, while ${\bm P}$ typically specifies the
format of the grid and the format of the thread blocks.
%(see Section~\ref{sec:embodiment} for further details of 
%this technique applied to \cuda).
%%


Let ${\cal E}$ be a high-level performance metric (running time, memory consumption)
for ${\cal P}$ that we want to
optimize.  More precisely, given the values of the data parameters
${\bm D}$,  our goal is to find values of the program parameters
${\bm P}$ such that the execution of ${\cal P}$ optimizes
${\cal E}$.  
%\toremove{Here, optimizing could mean maximizing, as in the case of a
%performance metric such as hardware occupancy, or minimizing, as in the case
%of a performance metric like execution time. }
Performance prediction models attempt to estimate ${\cal E}$ from a
combination of ${\bm P}$, ${\bm D}$, ${\bm H}$, and some model- or
platform-specific low-level metrics ${\bm L} = (L_1, \ldots,
L_\ell)$ (memory bandwidth, cache miss rate, etc.).
It is natural to assume that these low-level performance
metrics are themselves functions of ${\bm P}$, ${\bm D}$, ${\bm
  H}$.  This is an obvious observation from models based on PRAM such
as TMM~\cite{DBLP:journals/fgcs/MaAC14} and
MCM~\cite{DBLP:conf/parco/HaqueMX15}.
%%

Therefore, we look to obtain values for these low- and high-level metrics
given values for program, and data parameters.
To address our optimization goal, we  use the following strategy.
At the compile-time of program ${\cal P}$,  for each metric, 
we determine a mathematical formula expressing that metric as a  
function of the data and program parameters.
This mathematical formula takes the 
form of what we call a {\em rational program}.
%with some selected values of $D_1, \ldots, D_d$ which do not need to be the ones that  ${\cal P}$ will be executed on.
At the runtime of ${\cal P}$, given specific values of ${\bm D}$
and a choice of ${\bm P}$, we can evaluate the rational program to
obtain a value for each metric and thus for ${\cal E}$.  Repeating
this for all possible choices of ${\bm P}$ (assumed to be finite
in number) yields values of ${\bm P}$
optimizing ${\cal E}$.  This strategy is detailed in
Section~\ref{sec:steps}, while Section~\ref{sect:ratprog} 
is dedicated to the notion of a rational program.

One could view a rational program as a computer program that,
for input values $x_1, \ldots, x_n$, computes and returns a value
$y = f(x_1, \ldots, x_n)$, where $f$ is a function in the sense of a
programming language, say C/C++.  However, a rational program is 
more than that, due to the process we use to determine $f$.

%% Specifically, the encoding of some model as a flow chart 
%% whose nodes can then be approximated as a rational function
%% is a powerful idea which can be used
%% to simplify models and extrapolate results.

%We  take the remainder of this section to describe the 
%ideas underpinning rational programs.

\section{Rational Programs}
\label{sect:ratprog}

Let $X_1, \ldots, X_n, Y$ be pairwise different
variables\footnote{Variables refer to both its mathematical meaning
  and programming language concept.}.  Let ${\cal S}$ be a sequence of
three-address code (TAC~\cite{Aho:1986:CPT:6448}) instructions such
that the variables occurring in ${\cal S}$ that are never
assigned a value by an instruction
of ${\cal S}$ are exactly $X_1, \ldots, X_n$.


\begin{definition}
\label{defi:rationalprogram}
We say that the sequence ${\cal S}$ is a
{\em rational program} in $X_1, \ldots, X_n$ evaluating $Y$
if the following two conditions hold:
\begin{enumerateshort}
\item every arithmetic operation used in ${\cal S}$ is either an addition,
  a subtraction, a multiplication, a division, 
  or a comparison (for equality or the natural order $\le$)
  %($=$, $<$),
  of two rational numbers numbers, in either fixed or arbitrary precision.
\item after specializing in ${\cal S}$ the variables $X_1, \ldots, X_n$ to
      rational numbers $x_1, \ldots, x_n$, the execution of the 
      specialized sequence always terminates and the last
      executed instruction assigns a rational number to $Y$.
\end{enumerateshort}
\end{definition}

It is worth noting that the above 
definition can easily be extended to include
Euclidean division, the integer part operations floor and ceiling,
and arithmetic over rational numbers.
For Euclidean division one can write a rational program
evaluating the quotient $q$ of integer $a$ by $b$, 
leaving the remainder $r$ to be simply calculated as $a - qb$.
Then, floor and ceiling can be computed via Euclidean division.
Rational numbers and their associated arithmetic are easily implemented using
only integer arithmetic.
Therefore, by adding these operations to 
Definition~\ref{defi:rationalprogram}, the 
class of rational programs does not change.
We regard rational programs as such henceforth.
%Thus, we add such operations to the definition
%of a rational program and adopt this definition henceforth.

\section{Rational Programs as Flowcharts}

For any sequence ${\cal S}$ of computer program 
instructions, one can associate  ${\cal S}$
with a {\em control flow graph} (CFG).
In the CFG of ${\cal S}$, the nodes are the {\em basic blocks} of ${\cal S}$.
\iffalse 
the sub-sequences of ${\cal S}$ such that 
\begin{enumerate}[(i)]
	\item  each instruction except the last one is not a branch, and
	\item which are maximum in length with property (i).
\end{enumerate}
Moreover, in the CFG of ${\cal S}$, 
there is a directed edge from a basic block $B_1$ to a basic block
$B_2$ whenever, during the execution of ${\cal S}$, one can jump from
the last instruction of $B_1$ to the first instruction of $B_2$.
%%
\fi
Recall that  a {\em flowchart} is another
graphic representation of a sequence of computer program  instructions.
In fact, CFGs can be seen as particular flowcharts.
%%

If, in a given flowchart ${\cal C}$, every arithmetic
operation occurring in every (process or decision) node is
either an addition, subtraction, multiplication, or comparison of integers
in either fixed or arbitrary precision
then ${\cal C}$ is the flowchart of a rational sequence of computer
program instructions.
Therefore, it is meaningful to depict rational programs
using flowcharts, and vice versa, 
flowcharts as rational programs.
For example, one could consider the metric 
of theoretical {\em hardware occupancy} as defined by 
NVIDIA. 
The following example details its definition, 
its depiction as a flowchart, and its dependency on
program, data, and hardware parameters.

\begin{example}
\label{ex:cuda}
{\em
The hardware {\em occupancy} is a measure of a program's effectiveness in using
the Streaming Multiprocessors (SMs) of a GPU.
It is calculated from a number of hardware parameters, namely:
\begin{itemizeshort}
\item[-] the maximum number $R_{\rm max}$ of registers per thread block,
\item[-] the maximum number $Z_{\rm max}$ of shared memory words per thread block,
\item[-] the maximum number $T_{\rm max}$ of threads per thread block,
\item[-] the maximum number $B_{\rm max}$ of thread blocks per SM and 
\item[-] the maximum number $W_{\rm max}$ of warps per SM,
\end{itemizeshort}
as well as low-level kernel-dependent performance metrics, namely:
\begin{itemizeshort}
\item[-] the number $R$ of registers used per thread and 
\item[-] the number $Z$ of shared memory words used per thread block,
\end{itemizeshort}
and a program parameter, namely the 
number $T$ of threads per thread block.
%%
The occupancy of a {\cuda} kernel is defined as
the ratio between the number of active warps
per SM and the maximum number of warps per SM, namely:
\begin{equation}
\label{eq:occupancy}
W_{\rm active} / W_{\rm max}, \ \ {\rm where} \ \ 
    W_{\rm active} \fixed{\leq}{Changed to an inequality to highlight the
    upper bound given by this equation: Fine.} \min \left( \lfloor B_{\rm active} T / 32 \rfloor, W_{\rm max}  \right)
\end{equation}
and $B_{\rm active}$ is given by the flowchart
in Figure~\ref{fig:occupancysimpleflowchart}.
%%
This flowchart shows
how one can derive a rational program computing
$B_{\rm active}$ from $R_{\rm max}$, $Z_{\rm max}$,
$T_{\rm max}$, $B_{\rm max}$, $W_{\rm max}$, $R$, $Z$, $T$.
%%
It follows from Formula (\ref{eq:occupancy})
that $W_{\rm active}$ can also be computed 
by a rational program from $R_{\rm max}$, $Z_{\rm max}$,
$T_{\rm max}$, $B_{\rm max}$, $W_{\rm max}$, $R$, $Z$, $T$.
%%
Finally, the same is true for the occupancy of a {\cuda} kernel
using $W_{\rm active}$ and $W_{\rm max}$.
}
\end{example}
\tikzset{%
	>={Latex[width=2mm,length=2mm]},
	% Specifications for style of nodes:
	base/.style = {rectangle, rounded corners, draw=black,
		minimum width=4cm, minimum height=1cm,
		text centered, font=\sffamily},
	activityStarts/.style = {base, fill=blue!30},
	startstop/.style = {base, fill=red!30},
	activityRuns/.style = {base, fill=green!30},
	process/.style = {base, minimum width=2.5cm, fill=blue!15,
		font=\ttfamily},
	ifstatement/.style = {base, diamond, aspect=2.5, scale=0.7, font=\Large},
}
\usetikzlibrary{shapes.geometric}
\begin{figure}
	\centering
\begin{tikzpicture}[node distance=2.5cm,scale=.5, 
    every node/.style={fill=white,scale=.7, font=\sffamily}, align=center]
  % Specification of nodes (position, etc.)
  
  \node (activemax) [ifstatement] {$T \, B_{max} \leq 32 \, W_{max}$ and\\ $R \,  T \,  B_{\rm max} \leq R_{\rm max}$ and\\ $Z \,  B_{\rm max} \leq Z_{\rm max}$?};
  
  \node (activewmax) [ifstatement, below of=activemax, yshift=-9.5em] {$32 \,  W_{\rm max} \leq T \,  B_{\rm max}$, and\\ $32 \,  W_{\rm max} \,  R \leq R_{\rm max}$ and\\ $32 \,  W_{\rm max} \,  Z \leq Z_{\rm max} \,  T$?};
  
  \node (activermax) [ifstatement, below of=activewmax, yshift=-9.5em] {$R_{\rm max} \leq R \,  T \,  B_{\rm max}$ and\\ $R_{\rm max} \leq R \,  32 \,  W_{\rm max}$ and\\ $R_{\rm max} \,  Z \leq R \,  T \,  Z_{\rm max}$?};

  \node (activezmax) [ifstatement, below of=activermax, yshift=-9.5em] {$Z_{\rm max} \leq B_{\rm max} \,  Z$ and\\ $Z_{\rm max} \,  T \leq 32 \,  W_{\rm max} \,  Z$ and\\ $Z_{\rm max} \,  R \,  T \leq Z \,  R_{\rm max}$?};

  \node (bmax) [process, right of=activemax, xshift=16em, text width=12em] {$B_{active} = B_{max}$};
  \node (wmax) [process, right of=activewmax, xshift=16em, text width=12em] {$B_{\rm active} =  \lfloor( 32 \,  W_{\rm max})/T  \rfloor$};
  \node (rmax) [process, right of=activermax, xshift=16em, text width=12em] {$B_{\rm active} =  \lfloor( R_{\rm max}/(R \,  T) \rfloor$};
  \node (zmax) [process, right of=activezmax, xshift=16em, text width=12em] {$B_{\rm active} =  \lfloor Z_{\rm max}/Z \rfloor$};
      
  \node (fail) [startstop, below of=activezmax, yshift=-3.5em] {$B_{\rm active} = 0$
(Failure to Launch)};

  \draw[->] (activemax) -- node[] [yshift=0.45em]{No} (activewmax);
  \draw[->] (activewmax) -- node[] [yshift=0.45em]{No} (activermax);
  \draw[->] (activermax) -- node[] [yshift=0.45em]{No} (activezmax);
  \draw[->] (activezmax) -- node[] [yshift=0.45em]{No} (fail);  
  
  \draw[->] (activemax) -- node[] [xshift=-0.5em]{Yes} (bmax);
  \draw[->] (activewmax) -- node[] [xshift=-0.5em]{Yes} (wmax);
  \draw[->] (activermax) -- node[] [xshift=-0.5em]{Yes} (rmax);
  \draw[->] (activezmax) -- node[] [xshift=-0.5em]{Yes} (zmax);
  
  \draw[->]  ([shift=({0,2.5em})]activemax.north) to (activemax.north);
\end{tikzpicture}
\caption{Rational program (presented as a flow chart) for 
the calculation of the number of active blocks per streaming processor
in a {\cuda} kernel.}
\label{fig:occupancysimpleflowchart}
\end{figure}




\section{Piece-Wise Rational Functions in Rational Programs}
\label{sec:prf_rp}

We begin with an observation describing the fact that a rational program
can be viewed as a piece-wise rational function \footnote{Here, rational function is in the
	sense of algebra, see Section~\ref{sec:paramest}.} .
\begin{observation}
\label{obs:rationality}
{\em
Let ${\cal S}$ be a rational program in $X_1, \ldots, X_n$ evaluating $Y$.
%%
Let $s$ be any instruction of ${\cal S}$ other than a branch or an
integer part instruction.  
Hence, this instruction can be of the form
$C = A + B$, $C = A - B$, $C = A \times B$, where $A$ and
$B$ can be any rational number.
%
Let $V_1, \ldots, V_v$ be the variables that are defined
at the entry point of the basic block of the instruction $s$.
An elementary proof by induction yields the following fact.
There exists a rational function in $V_1,
\ldots, V_v$ denoted $f_s(V_1, \ldots, V_v)$
such that $C = f_s(V_1, \ldots, V_v)$ for all possible values of
$V_1, \ldots, V_v$.
%
From there, one derives the following observation.
There exists a partition ${\cal T} = \{ T_1, T_2, \ldots    \}$ 
of ${\Q}^n$ (where ${\Q}$ denotes the field of rational numbers)
and rational functions $f_1(X_1, \ldots, X_n)$, 
$f_2(X_1, \ldots, X_n)$, $\ldots\ $
such that, if $X_1, \ldots, X_n$ receive respectively 
the values $x_1, \ldots, x_n$, then
the value of $Y$ returned by ${\cal S}$ is one of
$f_i(x_1, \ldots, x_n)$ such that
$(x_1, \ldots, x_n) \in T_i$ holds.
%%
In other words, ${\cal S}$ computes $Y$ as a
{\em piece-wise rational function} \, (PRF).
Notice if ${\cal S}$ contains
only one basic block then ${\cal S}$ 
can be trivially given by a single rational function.

Example~\ref{ex:cuda} 
shows that the hardware occupancy of a {\cuda}
kernel is given as a piece-wise rational function
in the variables $R_{\rm max}$, $Z_{\rm max}$,
$T_{\rm max}$, $B_{\rm max}$, $W_{\rm max}$, $R$, $Z$, $T$.
Hence, in this example, we have $n = 8$, 
and, as shown by Figure~\ref{fig:occupancysimpleflowchart}, 
its partition of ${\Q}^n$ contains 5 parts
as there are 5 terminating nodes in the flowchart.

}
\end{observation}

Suppose that a flowchart ${\cal C}$ 
representing the rational program ${\cal R}$
is partially known; to be precise, suppose that the decision nodes
are known (that is, the mathematical expressions
defining them are known) while the process nodes
are not. 
Then, from Observation~\ref{obs:rationality},
each process node can be given by
a one or more rational functions.
Trivially, a single formula can also
be seen as a flowchart with a single process node.
Determining each of those rational functions
can be achieved by solving an \textit{interpolation}
or \textit{curve fitting} problem.
More generally, if the sequence of instructions in 
a process node involves non-rational functions (e.g. $\log$)
we can apply Stone-Weierstrass Theorem~\cite{stone1948generalized}
to approximate each of those by a PRF.


It then follows that any performance metric,
which can be depicted as a flow chart
or a formula,
can also be represented as a piece-wise rational function, 
and thus a rational program. 
For high-level performance metrics, which relies on low-level metrics,
one could work recursively, first determining rational programs
for the low-level metrics which depend on $\bm{P}$, $\bm{D}$, and $\bm{H}$,
and then constructing a rational program for the high-level metric 
from those rational programs.
Hence, by this recursive construction, we can fully
determine a rational program for 
a high-level metric depending only on $\bm{P}$, $\bm{D}$, and $\bm{H}$.
\fixed{Of course, hardware parameters could be fixed given a target architecture
to yield a rational program which depends only on $\bm{P}$ and $\bm{D}$.}{ And then what?}
\fixed{Again, notice that even where formulas for low-level metrics are not known,
it is still possible to estimate them as PRFs, and thus rational programs,
via a curve fitting.}{Haven't we said that already. Yes, but I believe this is repetition 
is for emphasis that maybe some readers may miss when skimming through a "theory" section.}


As an example, consider occupancy (Example~\ref{ex:cuda}).
One could first determine PRFs for the number of
registers user per thread and the amount of shared memory
used per thread block. Then, a PRF is determined
for the number of active blocks (Figure~\ref{fig:occupancysimpleflowchart})
from these two low-level metrics, and a few more hardware
and program parameters. 
Thus, by recursive construction, 
we have a PRF depending only on 
program and hardware parameters.

%Recall that building such a rational
%program ${\cal R}$ is our goal.
%Once ${\cal R}$ is known, it can be
%used at runtime (that is, when the program ${\cal P}$ is run
%on specific input data) to compute optimal values for the program
%parameters. This is exactly what is achieved in our tool
%implementing this technique. 
%Before exploring this tool
Lastly, we make one final remark. 
We assumed that the decision nodes in the flowchart of the rational program were known, 
however, we could relax this assumption.
Indeed, each decision node is given by
a series of rational functions.
Hence, those could also be determined by
solving curve fitting problems.
However, we do not discuss this further
since it is not needed in our proposed
technique or implementation presented in the remainder of this thesis.
