\begin{example}
\label{ex:cuda}
{\em
The hardware {\em occupancy} is a measure of a program's effectiveness in using
the Streaming Multiprocessors (SMs) of a GPU.
It is calculated from a number of hardware parameters, namely:
\begin{itemizeshort}
\item[-] the maximum number $R_{\rm max}$ of registers per thread block,
\item[-] the maximum number $Z_{\rm max}$ of shared memory words per thread block,
\item[-] the maximum number $T_{\rm max}$ of threads per thread block,
\item[-] the maximum number $B_{\rm max}$ of thread blocks per SM and 
\item[-] the maximum number $W_{\rm max}$ of warps per SM,
\end{itemizeshort}
as well as low-level kernel-dependent performance metrics, namely:
\begin{itemizeshort}
\item[-] the number $R$ of registers used per thread and 
\item[-] the number $Z$ of shared memory words used per thread block,
\end{itemizeshort}
and a program parameter, namely the 
number $T$ of threads per thread block.
%%
The occupancy of a {\cuda} kernel is defined as
the ratio between the number of active warps
per SM and the maximum number of warps per SM, namely:
\begin{equation}
\label{eq:occupancy}
W_{\rm active} / W_{\rm max}, \ \ {\rm where} \ \ 
    W_{\rm active} \fixed{\leq}{Changed to an inequality to highlight the
    upper bound given by this equation: Fine.} \min \left( \lfloor B_{\rm active} T / 32 \rfloor, W_{\rm max}  \right)
\end{equation}
and $B_{\rm active}$ is given by the flowchart
in Figure~\ref{fig:occupancysimpleflowchart}.
%%
This flowchart shows
how one can derive a rational program computing
$B_{\rm active}$ from $R_{\rm max}$, $Z_{\rm max}$,
$T_{\rm max}$, $B_{\rm max}$, $W_{\rm max}$, $R$, $Z$, $T$.
%%
It follows from Formula (\ref{eq:occupancy})
that $W_{\rm active}$ can also be computed 
by a rational program from $R_{\rm max}$, $Z_{\rm max}$,
$T_{\rm max}$, $B_{\rm max}$, $W_{\rm max}$, $R$, $Z$, $T$.
%%
Finally, the same is true for the occupancy of a {\cuda} kernel
using $W_{\rm active}$ and $W_{\rm max}$.
}
\end{example}