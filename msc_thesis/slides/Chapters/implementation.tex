\section{Implementation of KLARAPTOR}

\begin{frame}[fragile]{Annotations and preprocessing source code}
    \begin{block}{}
        \begin{minted}{c}
        
#pragma kernel_info_size_param_idx_Sample = 1; 
#pragma kernel_info_dim_sample_kernel = 2;

__global__ void Sample (int *A, int N) { 
    int tid_x = threadIdx.x + blockIdx.x*blockDim.x;
    int tid_y = threadIdx.y + blockIdx.y*blockDim.y;
    ...
}
        \end{minted}
        \begin{itemize}
            \item Annotating and preprocessing the source code makes it compatible with the KLARAPTOR tool, 
            enabling the automatic determination of optimal kernel launch parameters.
            \item CUDA program should target at least CUDA Compute Capability 7.5, no CUDA runtime API calls, 
            and block and grid dimensions must be declared as dim3 structs.
            \item Add two pragmas for each kernel, specifying kernel dimension and the 
            index of the kernel input argument corresponding to the data size N
        \end{itemize}
    \end{block} 
\end{frame}

\begin{frame}{Input/Output builder}
    \begin{block}{}
        The Input/Output Builder Pass creates an "instrumented binary" that embeds an IO mechanism to communicate with the 
        helper program for each kernel invocation, obtaining optimal kernel launch parameters at runtime.
        \begin{itemize}
            \item (i) Obtain the LLVM intermediate representation (IR) of the instrumented source code and find all CUDA driver 
            API kernel calls.
            \item (ii) Using the annotated information for each kernel, determine which variables in the IR contain the value 
            of N for a corresponding kernel call.
            \item (iii) Insert a call to an IO function immediately before each kernel call to pass the runtime value of N to 
            the corresponding helper program and retrieve the optimal kernel launch parameters.
        \end{itemize}
    \end{block}
\end{frame}

\begin{frame}{Building a helper program: data collection}
    \begin{block}{}
        Data collection is necessary to perform rational function approximation and gather metrics on performance 
        counters and runtime metrics.
        \begin{itemize}
            \item (i) Architecture-specific performance counters influenced by the Compute Capability (CC) of the device.
            \item (ii) Runtime-specific performance counters that depend on the kernel's behavior, including memory access 
            patterns and the number of executed warps.
            \item (iii) Device-specific parameters describing the actual GPU card, including memory bandwidth, departure delay for 
            memory accesses, number of SMs, clock frequency, and instruction delay.
        \end{itemize}
    \end{block}
\end{frame}

\begin{frame}[fragile]{Building a helper program: data collection}
    \begin{block}{Runtime-specific performance counters}
        {
            \begin{tcolorbox}
                \scriptsize
                \begin{verbatim}
[trace: n=4096, bx=32, by=8, elapsed_Convolution2D_kernel=219.9013 (ms)] ... PASS
==PROF== Disconnected from process 41150
"Metric Name","Metric Unit","Metric Value"
"dram__bytes_read.sum","byte","68,456,288"
"dram__bytes_write.sum","byte","66,864,032"
"l1tex__t_bytes_pipe_lsu_mem_global_op_ld.sum.per_second","byte/second","1,293,146,285,512.62"
"l1tex__t_bytes_pipe_lsu_mem_global_op_st.sum.per_second","byte/second","123,294,394,447.39"
"l1tex__t_sectors_pipe_lsu_mem_global_op_ld.sum","sector","21,984,780"
"l1tex__t_sectors_pipe_lsu_mem_global_op_st.sum","sector","2,096,128"
"launch__block_dim_x","block","32"
"launch__block_dim_y","block","8"
"launch__block_dim_z","block","1"
"launch__grid_dim_x","","128"
"launch__grid_dim_y","","512"
"launch__grid_dim_z","","1"
"launch__registers_per_thread","register/thread","30"
"launch__shared_mem_per_block_dynamic","byte/block","0"
"launch__shared_mem_per_block_static","byte/block","0"
"sm__warps_active.avg.pct_of_peak_sustained_active","%","87.46"
"smsp__average_inst_executed_per_warp.ratio","inst/warp","46.98"
"smsp__inst_executed.avg.per_cycle_active","inst/cycle","0.18"
"smsp__inst_executed_op_global_ld.sum","inst","4,716,288"
"smsp__inst_executed_op_global_st.sum","inst","524,032"
\end{verbatim}
\end{tcolorbox}
        }
    \end{block}
\end{frame}

\begin{frame}[fragile]{Building a helper program: data collection}
    \begin{block}{Device-specific parameters}
        {
            \begin{tcolorbox}
                \scriptsize
                \begin{verbatim}
[Issue_cycles: 1]
[Mem_bandwidth: 448.06]
[Mem_LD: 501]
[Departure_del_uncoal: 2]
[Departure_del_coal: 2]
[Active_SMs: 40]
[Freq: 1785]
[Load_bytes_per_warp: 128]

[device_name: NVIDIA GeForce RTX 2070 SUPER]
[driver_version: 11.7]
[runtim_version: 11.7]
[compute_capability: 7.5]
[global_memory_bytes: 8361803776]
[n_cores_per_sm:  64]
[n_blocks_per_sm: 32]
[n_cores: 2560]
[memory_clock_rate_ghz: 7.00]
[memory_bus_width_bits: 256]
[l2_cache_size_bytes: 4194304]
[total_constant memory_bytes: 65536]
[total_shared_memory_per_block_bytes: 49152]
[total_registers_available_per_block: 65536]
[warp_size: 32]
                \end{verbatim}
            \end{tcolorbox}
        }           
    \end{block}
\end{frame}

\begin{frame}[fragile]{Building a helper program: outlier removal}
    \begin{block}{}
        \begin{itemize}
            \item Outlier removal addresses potential noise in the empirical data gathered from NVIDIA’s Nsight Compute 
            (ncu) and enhances the accuracy of the parameter estimation process.
            \begin{itemize}
                \item (i) Profile the program for small input sizes and obtain MWP-CWP estimation for clock-cycles 
                for various thread block configurations.
                \item (ii) Calculate Q1, Q3, and IQR for the estimated clock-cycles and determine the upper inner fence.
                \item (iii) Identify and remove outliers exceeding the upper inner fence threshold.
            \end{itemize}
        \end{itemize}
        {
            \begin{tcolorbox}
                \scriptsize
                \begin{verbatim}
128 1 32 72869.600000
128 1 64 78474.600000
128 1 128 89684.600000
256 1 64 294064.600000
256 1 32 361929.200000
512 1 32 1271232.200000
512 1 64 1452818.400000
1024 1 64 4729979.800000
1024 1 32 4730039.600000
2048 1 32 18771193.800000
2048 1 64 18953199.200000
2048 1 512 18953199.200000
2048 1 1024 18953199.200000
                \end{verbatim}
            \end{tcolorbox}
        }
    \end{block}
\end{frame}

\begin{frame}[fragile]{Building a helper program: rational function approximation}
    \begin{block}{}
        \begin{itemize}
            \item The goal of this step is to determine rational functions that estimate low-level performance metrics or other intermediate values in the rational program ${\cal R}$. These rational functions are used to build the objective function ${\cal E} = f({\bm P}, {\bm D}, {\bm H})$
            \item A rational function is a fraction of two polynomials, with defined degree bounds on each variable, 
            the polynomials can be represented by coefficients that we aim to estimate: 
            {
                \small
                    \begin{align}
                    f(X_1,\dots,X_n) &= \frac{\alpha_1\cdot(X_1^0\cdots X_n^0) \;+\; 
                        \dots \;+\;  \alpha_i\cdot(X_1^{u_1}\cdots X_n^{u_n})}{\beta_1\cdot(X_1^0\cdots X_n^0) \;+\;
                        \dots \;+\; \beta_j\cdot(X_1^{v_1}\cdots X_n^{v_n})}
                    \end{align}
            }
            \item We perform parameter estimation on the coefficients of the polynomials in the rational function. 
            The estimation process involves solving an over-determined system of linear equations.
            \item We use numerical analysis techniques such as singular value decomposition (SVD) and linear least squares 
            to solve the system.
        \end{itemize}
    \end{block}
\end{frame}